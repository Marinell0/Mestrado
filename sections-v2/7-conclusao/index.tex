\section{Conclusão}

Este trabalho apresenta uma implementação com novas funcionalidades focadas em melhorar a comunicação entre fluxos de trabalho e obtenção de dados de maneira rápida e eficiente dentro de LIMS que utilizam BPM: Workflows dinâmicos e compartilhamento de atividades entre workflows. Anteriormente, o sistema só poderia seguir a ordem das atividades como modelado inicialmente. Agora os workflows podem ser alterados para que a visualização seja relevante para o usuário que o está utilizando.

O compartilhamento de mensagens entre workflows também acelerou o acesso a informações entre times diferentes dentro de um mesmo laboratório que utiliza o LIMS. Atividades podem ficar dependentes de um outro fluxo de trabalho (conserto de equipamento, por exemplo) e a execução do mesmo pelo outro time de usuários disponibiliza novas atividades. Isso ajuda a diminuir a quantidade de erros que pode ocorrer no sistema e melhora a integração do laboratório às regras regulamentadoras que podem existir para determinado workflow.

Os workflows dinâmicos permitem que os usuários tenham uma interação completamente personalizada para o momento em questão, mudando a modelagem BPM de forma a encontrar informações e executar os seus trabalhos de maneira muito mais rápida e eficiente, alterando a interface do LIMS para que uma atividade em específica do workflow seja focada.
Essa centralização em uma atividade selecionada resulta em maior eficiência e flexibilidade no gerenciamento do projeto, facilitando o uso do software LIMS tanto para gerentes de laboratório quanto para técnicos.

No contexto do Big Data, workflows dinâmicos facilitam a visualização e a integração das informações no sistema quando o volume de dados é muito grande, permitindo que o usuário visualize os dados de forma mais intuitiva e compreensível por meio da disponibilização visual personalizada, auxiliando na interpretação dos dados e na tomada de decisões.

A combinação de múltiplos workflows para pertencerem ao mesmo modelo BPM e compartilhar atividades possibilita que times diferentes dentro de um mesmo laboratório troquem informações entre si, permitindo o envio de notificações quando uma atividade for concluída, ou realizar o pedido de aprovação de uma atividade a um supervisor. Um médico pode fazer um pedido de exame ao laboratório do hospital, executando a atividade que notificará ao técnico de laboratório. O técnico irá aprovar ou reprovar a atividade, fazer todos os passos necessários e apenas o resultado do pedido de exame irá ser disponibilizado ao médico, tendo dois workflows unidos trocando informações entre diferentes usuários.

Este tipo de integração também pode ser utilizada com a implementação de workflows dinâmicos, deixando as implementações ainda mais eficientes. O compartilhamento de atividades dentro de um workflow facilita a acessibilidade de gerentes de laboratório às atividades de todo o processo que ocorre dentro da organização, dando a habilidade de alterar a visualização como um todo, remontando até os workflows unidos pela atividade compartilhada.

Essas implementações podem ser utilizadas em todos os workflows. Os workflows dinâmicos podem ser utilizados em modelos já implementados sem nenhuma alteração, mas levar em consideração que as atividades podem ser centralizadas pode ajudar na modelagem do BPM.

O compartilhamento de atividades pode ser implementado em novos workflows ou com adaptação de workflows já existentes, devendo ser remodelado para que haja a comunicação entre diferentes usuários do LIMS.

Foram realizados testes nos workflows BPL, CTTX e CENTRARE. Para os workflows BPL e CTTX, foi utilizada a implementação de compartilhamento de atividades, unindo múltiplos workflows e compartilhando informações entre eles. Para a implementação de workflows dinâmicos, foi utilizado o workflow CENTRARE pela sua complexidade e quantidade de atividades que devem ser executadas por diferentes usuários.

Foi feito com sucesso a implementação da funcionalidade no sistema LIMS Flux, que utiliza como base de montagem de workflows o Business Process Model (BPM), permitindo que administradores do Flux montassem workflows já com estas funcionalidades em mente para disponibilização de informações ao usuário, e também permitindo que usuários de workflows do Flux obtivessem informações com muito mais velocidade, aumentando a eficiência dos trabalhos realizados.

Como o sistema Flux utiliza de instâncias para separações dos workflows por execução, o compartilhamento de informações entre atividades foi feito para diferentes instâncias, ou seja, BPMs com construções diferentes podem comunicar entre execuções específicas de um workflow, ligando suas atividades onde necessário e, com isso, obtendo a funcionalidade de comunicação assíncrona entre usuários do sistema executando atividades que ficam disponíveis para outros.

Com isso, podemos unificar diferentes tipos de workflow utilizados em uma organização em um único workflow com múltiplas atividades iniciais. Assim, o workflow fica mais acessível para diferentes funcionários como gestor de laboratório, podendo ele acessar qualquer parte dos múltiplos workflows agrupados que fazem parte do laboratório e fazer a coleta e análise dos dados inseridos por outros funcionários, tudo em uma mesma tela.

A funcionalidade de workflows dinâmicos também funciona com a troca de informações, disponibilizando todas as informações de todas as instâncias compartilhadas quando a ordem das atividades é alterada, alterando o BPM para que o foco da execução atual seja na atividade selecionada.

Seguindo a utilização do Flux e a melhoria que obtivemos com workflows dinâmicos alterando a interface do usuário para que ela seja mais intuitiva e eficiente, temos que os resultados obtidos através de avaliação empírica foram muito positivos e que plataformas que implementam LIMS com BPM podem se beneficiar muito com este tipo de implementação. Para validar que haverá uma melhoria de eficiência quando implementado em um ambiente laboratorial real, é necessário fazer análises na velocidade de execução dos usuários.

Hoje o Flux já salva variáveis de horário de execução e acesso dos usuários às atividades, que poderiam ser utilizados para realizar um estudo estatística na velocidade dos trabalhos na utilização do sistema antes do recuso ser implementando e depois, fazendo a comparação de velocidade na execução dos trabalhos dos usuários que o utilizam.

Ainda existem problemas a serem atacados a partir deste trabalho, como a alteração da interface para melhor a identificação de pais diretos da atividade trocada e informações sobre compartilhamento de atividades, como quando atividades podem ser compartilhadas entre instâncias de mesmo tipo (hoje isso não é permitido: Pedidos de exame não são compartilhados entre laboratórios).

A árvore de atividades anteriores à atividade focada atualmente é necessária para garantir que todas as informações estejam disponíveis para execução de atividades futuras, mas quais atividades podem ser ofuscadas para garantir uma interface mais limpa e intuitiva pode ser estudado, já que não é possível saber quais informações serão importantes em um BPM genérico.

Além disso, é necessário realizar um estudo mais aprofundado sobre o compartilhamento de atividades com instâncias específicas, a fim de compreender de maneira abrangente com quais instâncias uma atividade pode ser compartilhada. Dessa forma, será possível determinar se as atividades podem ser compartilhadas com instâncias do mesmo tipo ou se a atividade deve conter propriedades que indiquem quais instâncias podem receber a informação, em vez de deixar essa escolha a critério do usuário. Essa análise detalhada tem o objetivo de automatizar o processo e torná-lo mais eficiente.