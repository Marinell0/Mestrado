\subsection{Proposta}

\subsubsection{Troca de ordem de atividades}

% Troca de ordem de atividades

% O que isso traz para os BPMs

% Isso já é possibilitado pelos BPMs?

Isso já pode ser feito com BPMs na parte de modelagem do mesmo, mas com grandes ressalvas nas partes de segurança, automação e eficiência do mesmo, já que as atividades do BPM podem ter dependências entre si. Isso já é tratado pelo Flux, disponibilizando as informações de maneira a apenas acelerar a visualização das variáveis pelo usuário

% O que isso traz para os usuários

A reordenação do workflow acelera o acesso do usuário a partes que realmente importam para ele no momento, alterando a interface já existente para acatar às necessidades do usuário.

% O que isso traz para os LIMS

Como os LIMS tem o grande problema de ter uma interface complexa, com muitos dados sendo mostrados na tela e uma dificuldade dos usuários a se adaptar e utilizar o LIMS \R, com a alteração da interface para mostrar ao usuário a atividade e os dados requisitados, removemos complexidade na busca de informações dentro do software, aumentando a eficiência na busca de informações.

% Porque isso é necessário?

Isso é necessário pois a premissa de um LIMS que utiliza BPM dentro de uma área laboratorial é a otimização de processos de negócio e aumento de eficiência dos trabalhos. Com a alteração da interface, podemos obter este resultado.

%%%%%%%%%%%%%%%%%%%%%%%%%%%%%%%%%%

\subsubsection{Múltiplos pontos de início de um workflow}

% Separação de setores do workflow em categorias de usuários

% O que isso traz para os BPMs?

% Isso já é possibilitado pelos BPMs?

Nos BPMs, não existem múltiplas atividades iniciais, já que um processo consiste de um início, e só a partir dele temos como dar progresso ao modelo de negócios. Com isso, a adição de múltiplas atividades iniciais que podem ser iniciadas por pessoas diferentes adiciona uma poderosa ferramenta a modelagem dos workflows.

% O que isso traz para os usuários

Isso facilita a intercomunicação de processos que compartilham de uma mesma ordem de atividades ou utilizam de informações do fluxo de atividades para continuar seu processo. Um exemplo disso é em um pedido de exame de sangue de um médico para um laboratório, que irá examinar a amostra e dar um resultado ao médico. Este médico só poderá continuar o processo de atendimento quando o exame tiver terminado, tendo assim a troca de informações entre médico e técnico de laboratório e melhorando a integração entre os dois processos.

% O que isso traz para os LIMS

A integração desta nova ferramenta a um LIMS melhora muito a eficiência e integração das atividades de um modelo de negócios com o outro, otimizando a troca de informações entre dois fluxos de trabalho. O médico pode receber uma notificação quando o exame ficar pronto, um técnico recebe notificação quando uma amostra chega e pode até ser feita a otimização deste caminho, com a amostra indo direto para uma máquina analisar.

Isso também facilita a utilização do software pelos usuários, já que fica mais intuitivo onde e quando o usuário deve entrar dentro da interface para acessar os dados que são de seu interesse.

% Porque isso é necessário?