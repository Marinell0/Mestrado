\subsection{Proposta}

Workflows dinâmicos tem como objetivo aumentar a velocidade do acesso aos dados dando a possibilidade ao usuário de selecionar uma atividade a ser focada como atividade principal no momento do acesso. Essa implementação muda a ordem do BPM modelado, permitindo que a atividade selecionada vire a primeira atividade do BPM, ou seja, essa será a primeira atividade que o usuário irá visualizar quando acessar o workflow.

Isso permite que o usuário visualize a atividade com as informações de importância pontual, pulando atividades que não contêm informações necessárias no momento. Com o foco da visualização na atividade selecionada, também há a disponibilização de dados antigos para que, na necessidade de algum dado anterior, ele ainda esteja disponível para acesso.

\subsubsection{Troca de ordem de atividades}

% Troca de ordem de atividades

% O que isso traz para os BPMs

% Isso já é possibilitado pelos BPMs?

A possibilidade de alteração no foco de atividades não é possível com BPMs, pois a modelagem feita necessita que ela siga uma ordem fixa do processo de negócios para que não ocorra nenhum erro de dependência de atividades. Para isso, o Flux utiliza do BPM para saber quais atividades estão disponíveis para serem focadas e, assim, caso selecionadas, alterar o BPM para que a atividade inicial do BPM seja a atividade focada, alterando a ordem de execução do BPM mas ainda mantendo as dependências entre atividades.

% Isso já pode ser feito com BPMs na parte de modelagem do mesmo, mas com grandes ressalvas nas partes de segurança, automação e eficiência, já que as atividades do BPM podem ter dependências entre si. Isso já é tratado pelo Flux, disponibilizando as informações de maneira a apenas acelerar a visualização das informações pelo usuário

% O que isso traz para os usuários

A reordenação do workflow acelera o acesso do usuário a partes que realmente importam para ele no momento, alterando a interface já existente para acatar às necessidades do usuário.

% O que isso traz para os LIMS

Como os LIMS geralmente tem uma interface complexa, com muitos dados sendo mostrados na tela e uma dificuldade dos usuários a se adaptar e utilizar o LIMS~\cite{Tomlinson2022AOperations}, com a alteração da interface para mostrar ao usuário a atividade e os dados requisitados, removemos complexidade na busca de informações dentro do software, aumentando a eficiência na busca de informações.

% Porque isso é necessário?

Isso é necessário pois a premissa de um LIMS que utiliza BPM dentro de uma área laboratorial é a otimização de processos de negócio e aumento de eficiência dos trabalhos. Com a alteração da interface, podemos obter este resultado.
