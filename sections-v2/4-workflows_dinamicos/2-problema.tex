\subsection{Problema}

Embora os sistemas LIMS possam ser uma ferramenta valiosa para gerenciar e rastrear dados em laboratórios, eles podem apresentar alguns desafios, como a complexidade na sua utilização, com necessidade de quantidades significativas de conhecimento e treinamento~\cite{Avery2000ProductGuide.}.

Um dos grandes problemas que ele pode trazer para a área laboratorial ou médica é a falta de personalização: Para atender às necessidades de laboratórios específicos, os LIMS precisam ser altamente personalizáveis.

Os LIMS tem como finalidade serem uma ferramente para facilitar a coleta, armazenamento e análise de dados em laboratórios. Com esta capacidade, o LIMS permite que usuários tenham informações valiosas sobre experimentos, amostras e instrumentos utilizados na área em que está implementado.

O LIMS também pode ajudas a garantir a qualidade dos dados, melhorar a rastreabilidade e a conformidade regulatória e, com a disponibilização das informações em um ambiente centralizado, aumentar a cooperatividade entre equipes e departamentos.

A personalização da interface do sistema LIMS desempenha um papel crucial na facilitação da coleta e análise dos dados disponíveis. Com uma interface personalizada, os usuários podem ajustar o sistema para atender às suas necessidades específicas, melhorando a eficiência e a usabilidade do sistema.

Por exemplo, esta personalização pode disponibilizar informações relevantes naquele momento para o usuário que esteja procurando por dados específicos de um fluxo de trabalho que este gerencia.

Este recurso pode ajudar, também, a simplificar a navegação do sistema, permitindo que os usuários acessem rapidamente as informações que precisam. Isso pode ser particularmente útil para equipes que trabalham em vários projetos ou em diferente áreas de um laboratório, como no caso de um gerente laboratorial.