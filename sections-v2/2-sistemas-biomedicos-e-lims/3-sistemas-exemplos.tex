\subsection{Exemplos de sistemas utilizando BPM} \label{sec:lims-exemplo}

Nesta seção iremos falar sobre alguns LIMS que são conhecidos e citaremos algumas das suas capacidades para integração de um fluxo de trabalho dentro deles.

\subsubsection{Bika}

Bika é um LIMS gratuito que pode ser utilizado por qualquer organização. Além de ser um LIMS de graça, ele tem integração com vários equipamento biomédicos, mas ainda faltam interações entre equipamentos essenciais que fazem dele um LIMS menos poderoso por não conseguir se integrar com o fluxo de trabalho do laboratório de maneira generalizada \cite{Ademuyiwa2018DevelopmentBiobanking}.

O Bika permite que organizações configurem suas próprias práticas de laboratório, fluxos de trabalho, modelos de dados e relatórios de saída. O sistema é baseado em um modelo de informações flexível que pode ser adaptado às necessidades específicas de um laboratório.

Algumas das principais funcionalidades que se destacam são as de gerenciamento de amostras, gerenciamento de ensaios, gerenciamento de resultados e gerenciamento de clientes dentro do mesmo software.

A integração de equipamentos laboratoriais deve ser programada individualmente para cada equipamento pelos desenvolvedores do sistema, diminuindo a integração do software dentro da organização.

Além disso, a interface implementada para uma organização segue tópicos de preenchimento, não seguindo um fluxo de trabalho como um BPM modelado. Assim, seguir um passo a passo nesta plataforma fica com uma maior propensão ao erro, já que a plataforma deixa que o usuário execute qualquer tipo de atividade a qualquer momento.

\subsubsection{MetaLIMS}

O MetaLIMS é um sistema de gerenciamento de informações de laboratório (LIMS) que foi projetado para atender as necessidades de laboratórios de pesquisa e desenvolvimento (P\&D) em ciências da vida, química e materiais. Ele é desenvolvido pela MetaSystems, uma empresa que oferece soluções para laboratórios de genômica e citogenômica.

O sistema permite que os usuários registrem informações sobre amostras, experimentos, resultados, protocolos e instrumentos de laboratório em um único local centralizado. Ele também permite que os usuários rastreiem o status das amostras em tempo real, gerenciem tarefas e atribuam recursos, além de oferecer recursos de geração de relatórios e visualização de dados.

MetaLIMS utiliza conceitos de BPM (Business Process Management) para gerenciar os fluxos de trabalho e processos de laboratório. Ele permite que os usuários definam e gerenciem fluxos de trabalho personalizados para cada tipo de experimento ou análise, desde a solicitação da amostra até a geração de relatórios finais, definindo etapas sequenciais, atividades paralelas e condições de entrada e saída.

\subsubsection{Labvantage}

O Labvantage LIMS é um sistema de gerenciamento de informações de laboratório (LIMS) desenvolvido pela LabVantage Solutions, uma empresa de software que oferece soluções para laboratórios em diferentes setores. O sistema é projetado para atender às necessidades de laboratórios de diferentes tamanhos e complexidades em diversas indústrias, incluindo farmacêutica, biotecnologia, alimentos e bebidas, ambiental e química.

O LabVantage permite que os usuários configurem fluxos de trabalho de forma flexível, definindo etapas sequenciais, atividades paralelas e condições de entrada e saída por meio de definição do fluxo de trabalho utilizando BPM. Isso ajuda a garantir que todos os experimentos e análises sigam um processo padronizado e controlado, aumentando a qualidade e a confiabilidade dos dados gerados.

\subsubsection{Flux}

O LIMS Flux é um sistema de gerenciamento de informações de laboratório (LIMS) que permite que laboratórios gerenciem e rastreiem seus processos e dados de amostras. Ele será melhor explicado na seção \ref{sec:flux}

% Exemplo 1: MetaLIMS

% Exemplo 2: Labvantage

% Exemplo 3: Thermofisher