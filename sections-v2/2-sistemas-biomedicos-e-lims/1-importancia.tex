\subsection{Importância}

% Importância de sistemas biomédicos hoje

O grande volume de dados gerados e utilizados por aplicações e como analisá-los é tema de várias pesquisas, sejam elas por corporações ou por pesquisas acadêmicas \R. Principalmente na área biomédica, temos a geração de enormes quantidades de dados com a evolução das tecnologias de sequenciamento genético~\cite{luoJ2016}.

As áreas de telemedicina também estão sofrendo avanços enormes nos últimos anos com a vinda da pandemia do SarsCov2, o Covid19~\cite{bakhtiar2020, kronenfeld2021, GatesB.Colbert2020UtilityEra}, e com isso também houve o aumento exorbitante da geração de dados no campo médico para que houvesse um atendimento mais rápido e melhorado dos pacientes~\cite{MohdKhanapiAbdGhani2018PDFData, Coakley2015TransformingAnalytics}.

Essa quantidade de dados gerado deve ser armazenada, analisada e ter proteção contra acessos não autorizados. O software que obtiver essa quantidade de dados tem como responsabilidade assegurar todos os pontos anteriores, além de facilitar o acesso dos dados aos usuários por meio de interfaces facilitadoras.

Existem inúmeras soluções criadas para uso profissional, além de um grande número de pesquisas para tratar da crescente quantidade de dados gerados tanto por laboratórios quanto por sistemas de telemedicina~\cite{Mangrulkar2022AutomaticTechniques}.

Uma das soluções criadas para resolução deste problema é chamado de Laboratory information management system (LIMS), e já existem para uso pessoal ou de corporações, como o Bika~\cite{Goodblatt2006FosteringProcess}, MetaLIMS~\cite{Heinle2017MetaLIMSLabs}, Labvantage~\cite{Smallmon2017BiobankingSilos}, Flux~\cite{Melo2010SIGLa:Laboratories}, e suas funcionalidades serão explicados na seção~\ref{sec:lims-exemplo}
% Quantidade de dados sendo gerados pela área médica e laboratorial

% Soluções existentes no mercado hoje

% Introdução aos LIMS