\subsection{LIMS}

% O que são (de novo)

Laboratory information management system (LIMS), ou sistema laboratorial de manejamento de informação são softwares criados para gestão de dados, processos, máquinas e pessoas de forma segura e muitas vezes automatizada \cite{Stafford1998LIMS:Technology}.

% Onde podem ser utilizados

LIMS são utilizados principalmente na área laboratorial \R, mas podem ser utilizados em qualquer sistema que necessita de coleta de dados, automação e integração de pessoas e máquinas para obtenção de resultados consistentes e para manter a segurança de tais dados \R.

Um dos exemplos deste tipo de cenário é na área de telemedicina, que, com a vinda da pandemia do SarsCov2 no fim de 2019, sofreu um aumento significativo de sua utilização com mais e mais pacientes surgindo e com lockdowns acontecendo, fazendo com que pessoas ficassem em casa e não pudessem fazer visitas médicas presenciais \R.

Isso também levou ao grande aumento de dados sendo guardados por sistemas e, nos sistemas antigos presentes em muitos hospitais que ainda utilizam caneta e papel em grandes salas utilizadas apenas para armazenamento de dados de pacientes \R, esta quantidade de dados não poderia ser escalonada para abranger a quantidade requerida nos tempos da pandemia.

% Exemplos de utilização, tanto em áreas médicas ou laboratoriais

O próprio governo brasileiro expandiu a utilização de softwares no sistema único de saúde (SUS) \R, aumentando o projeto de unificação dos dados de pacientes por todo o país para que todos fossem armazenados em um banco que todos os médicos tivessem acesso para melhoria na eficiência de processos e unificação de hospitais, o  e-SUS, com enfase na "estratégia para reestruturar as informações da Atenção Primária em nível nacional" \R.

Para a área laboratorial, temos diversos exemplos de utilização para automação de processos e experimentos, além de aumento no nível de segurança de dados que é necessário para manter dados sigilosos apenas com as pessoas que podem acessá-los. \NO

Os LIMS são construídos seguindo uma estrutura de fluxo de trabalhos (workflows) \R, modelando como cada trabalho funciona dentro da área que estão implementados. \NO

% Quais categorias de pessoas são as melhores para utilizar o software (Manager, developer, médico, paciente...)

É importante que o LIMS esteja integrado com toda a atividade de onde está sendo utilizado para que aumente a eficiência e automação das atividades dentro do ambiente implementado. Isso inclui a utilização pelos funcionários, sendo o LIMS moldado para todo tipo de funcionário na companhia.

Uma das maneiras de estruturar estes fluxos de trabalho é com a utilização de Business Process Management (BPM).

BPM é uma metodologia de modelar trabalhos e processos que gera atividades representando fluxos de trabalho que são necessários para chegar a um objetivo dentro de uma organização \R. Com a modelagem do BPM, chegamos a uma arquitetura de processos que pode ser dividida de maneira a aumentar a eficacia de um grupo de pessoas no trabalho modelado \R.

% Utilização de BPMs no LIMS

Ao utilizar o BPM com LIMS, utilizamos o BPMN, ou Business Process Management Notation, uma notação que é padronizada \R e pode ser utilizada juntamente a um LIMS para modelar os processos e já poder integrar o fluxo de trabalho diretamente ao software de coleta de dados e automatização de processos \R.

% Beneficios de BPMs ( https://proceso.pro/en/blog/pros-and-cons-of-business-process-management-bpm/ )



A utilização de BPMs em LIMS ajuda na produtividade dos usuários, já que o sistema pode definir e automatiza partes do processo, reduzindo assim a quantidade de erros no processo por restringir as partes críticas das atividades seguindo as políticas do laboratório.
Com isso, temos também a redução de micro manejamento dos gerentes da empresa, por ter um controle sobre os dados maior com medidas de segurança e por todos os funcionários seguirem o mesmo protocolo de preenchimento de dados e acatamento de resoluções. \R 

% Maleficios de BPMs ( https://proceso.pro/en/blog/pros-and-cons-of-business-process-management-bpm/ )

Seguindo os princípios dos benefícios do BPM, os malefícios de sua integração seguem da falta de comunicação entre pessoas para seguir as regras da empresa caso os processos sejam modelados de maneira a não permitir a comunicação entre usuários do software, dificuldade em inovações caso a modelagem seja muito restritiva, não permitindo pensamentos criativos dentro da empresa.

Estes malefícios podem aparecer principalmente quando há uma abordagem inadequada ou excessivamente rígida na implementação da gestão de processos, e não necessariamente serão sempre um problema na organização que o implementa.

Para isto, é importante que tenham pessoas experientes com BPMs para implementação do mesmo dentro da organização, que demonstra mais um malefício dos BPMs: O custo elevado e a complexidade de implementação.

Para implementar corretamente um BPM, deve-se ter um alinhamento estratégico com os objetivos da organização, levando em consideração todas as partes que a compõem como, no caso de um laboratório, gerência, técnicos de laboratório, cientistas e estudantes.

% Beneficios de LIMS ( https://genemod.net/blog/lims-the-good-the-bad-and-the-ugly )

% Eficiencia, acuracia e produtividade

Os benefícios da utilização de um LIMS em um ambiente laboratorial vem do aumento da eficiência dos cientistas, técnicos, gerência e comunicação externa entre laboratórios vinda da integração do software com as partes necessárias para gravação e obtenção de dados. Sendo assim, um LIMS seguro pode fornecer todos os dados em uma tela rápida para a gerência e, caso um cientista vir a utilizar do software, ter outra tela mais relevante disponibilizando apenas as informações necessárias para aquele individuo, aumentando assim sua produtividade.

% Maleficios de LIMS  ( https://genemod.net/blog/lims-the-good-the-bad-and-the-ugly )

Os problemas que podem vir da implementação de um LIMS é que, se mal implementado, a interface pode ser não intuitiva, de forma a denegrir a produtividade dos envolvidos. Caso este LIMS não integre com as partes laboratoriais, ele pode se tornar uma dificuldade a mais no dia a dia do usuário, tendo que repetir dados em locais diferentes pela falta de automação, fazendo com que a acurácia também caia - erros de entrada de dados no programa podem ocorrer.

% Porque são importantes

% Como guardam os dados

Os dados de um LIMS precisam ser guardados em bancos de dados seguros e apenas liberar o acesso a esses dados para pessoas autorizadas, então é necessário que tenha um modelo de segurança de acesso integrado para que isso seja possível. isso é necessário \NO pois os dados que estão nestes LIMS, muitas vezes, são dados sensíveis e que não podem cair em mãos erradas \R \NO.

% Como lidam com segurança dos dados

Desta forma, é implementada uma hierarquia de usuários para que cada classe de usuário (Gerente, cientista, técnico...) tenha diferentes tipos de acesso aos dados.

% Benefícios de sua utilização

% Malefícios de sua utilização

% Arquitetura: Como são construídos (Seguindo que base)

% Como há a disponibilização de dados aos usuários

% Problemas atuais

% Beneficios que BPM trazem no LIMS

% Problemas que BPMs trazem no LIMS