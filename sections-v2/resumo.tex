\selectlanguage{Portuguese}

\begin{abstract}
    A necessidade de sistemas para coleta, armazenamento e análise de dados cresce a cada evolução tecnológica que surge com o tempo. A área biomédica e laboratorial não é exceção, e para isso existem sistemas especializados para este tipo de função: Os LIMS (Laboratory Information Management System).

    LIMS é um conceito de software que pode ser implementado de diversas formas, sendo uma delas o LIMS baseados em workflows, que são sistemas que utilizam do modelo de processo de negócios (BPM) de uma organização -representações visuais de como as atividades de negócio são realizadas - para melhorar a eficiência e otimizar os processos organizacionais. Estes tipos de sistemas tem como vantagem a disponibilização deste processo para que usuários possam executar seus trabalhos de forma eficiente, ordenada e com agilidade, mantendo as informações no sistema para que outros usuários possam acessar.

    Contudo, estes sistemas complexos armazenam todas as informações das organizações que o utilizam e, como são muitos dados sendo armazenados, o sistema precisa disponibilizá-los da maneira mais intuitiva possível.
    Este trabalho propõe implementações que alteram o modelo de processo de negócios construído no sistema para centralizar o foco do usuário em uma atividade, facilitando o acesso às informações requeridas no momento.

    Também foi implementado a habilidade de compartilhar informações entre fluxos de trabalho para que times possam se comunicar inteiramente pelo sistema, diminuindo a incidência de erros quando a comunicação entre equipes for necessária.

    Estas implementações estão em uso em diversos sistemas com ganhos para usuários. Em particular, citamos o workflow de Citotoxicidade (CTTX) que foi unificado com o workflow de Boas Praticas Laboratoriais (BPL) para compartilhamento de informações de equipamentos utilizados. Citamos também a visualização dinâmica do workflow CENTRARE, utilizado no Hospital da Baleia, para facilitar a pesquisa de informações dentro do workflow.

    Palavras-chave: LIMS; Sistemas de informação; BPM.
\end{abstract}
