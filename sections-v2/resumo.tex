\section{Resumo}

A necessidade de sistemas para coleta, armazenamento e análise de dados cresce a cada evolução tecnológica que surge com o tempo. A área biomédica e laboratorial não é exceção, e para isso existem sistemas especializados para este tipo de função: Os LIMS (Laboratory Information Management System).

LIMS baseados em BPMs (Business Process Model) são sistemas que modelam o fluxo de trabalho de uma organização e disponibilizam este processo para que usuários possam executar seus trabalhos de forma eficiente, ordenada e com agilidade, mantendo as informações no sistema para que outros usuários possam acessar.

Estes sistemas complexos armazenam todas as informações das organizações que o utilizam, precisando disponibilizá-los da maneira mais intuitiva possível. Para isso, foram criados implementações que alteram o modelo BPM construído no sistema para centralizar o foco do usuário em uma atividade, facilitando o acesso às informações requeridas no momento.

Também foi implementado a habilidade de compartilhar informações entre fluxos de trabalho para que times possam se comunicar inteiramente pelo sistema, diminuindo a incidência de erros quando a comunicação entre equipes for necessária.