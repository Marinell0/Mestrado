\subsection{Flux}

O LIMS Flux foi utilizado para implementação dos recursos de workflows dinâmicos e da agregação de múltiplos BPMs para troca de informações entre eles. Sua utilização foi feita pela facilidade na criação de workflows dentro do próprio software e a alta personalização disponibilizada.

Como estes recursos alteram como a visualização de um BPM ocorre, foram necessário ajustes para que a implementação de tais recursos fosse feita de maneira correta e consistente, sem que a alteração na visualização deixasse o LIMS com vulnerabilidades de segurança e que as informações estivessem sempre disponíveis quando necessário.

\subsubsection{Alterações feitas para workflows dinâmicos}

Para que o Flux pudesse focar em uma atividade dentro do workflow, alteramos a definição de instância do programa. Anteriormente, instâncias eram definidas como o inicio do fluxo de trabalho do usuário para execução do BPM.

Com a implementação de workflows dinâmicos, foi necessário a utilização de instâncias como o ponto de partida de execução do usuário naquele momento, e não necessariamente o inicio do fluxo de trabalho. Instancias ainda podem ser utilizadas seguindo a definição anterior caso o usuário não selecione nenhuma atividade para ser focada (Ou selecione a atividade inicial original).

No software, instâncias ainda são criadas apenas para a atividade original, mas por baixo dos panos, para disponibilizar as informações para o usuário de maneira conveniente, é criado uma instância temporária onde a atividade inicial é a atividade selecionada pelo usuário, disponibilizando esta atividade centralizada como ponto de partida para sua execução.

É importante que estas instâncias auxiliares não sejam salvas para que não haja duplicação de dados do workflow e para não ocorrer alterações no BPM original. A ordem das atividades sempre deve seguir a estrutura original do BPM para manter o fluxo desejado quando ele foi modelado.

Se a estrutura do BPM mudar de uma maneira a disponibilizar atividades antes de uma dependência, isso pode causar um fluxo de trabalho errôneo que não segue as restrições modeladas quando o workflow foi implementado no sistema, e isso não pode ocorrer para que o BPM ainda esteja em conformidade com as diretrizes do laboratório.

Para selecionar qual atividade inicial será utilizada na visualização atual, foi alterada a interface para adicionar um seletor de atividades, disponibilizando todas as atividades que o usuário tem permissão de acessar em uma lista ordenada pelo nome da mesma. Nele, temos o nome de todas as atividades do workflow selecionado que o usuário tem a permissão de visualizar.

Caso o usuário não selecione nenhuma atividade, a atividade inicial utilizada será a padrão, deixando o workflow na visualização padrão (Como podemos ver na figura \ref{fig:centrare_seletor_normal}). Quando o usuário seleciona uma atividade para ser focada, temos a alteração da interface para mostrar, como instância, todas as atividades executadas da atividade que o usuário selecionou no seletor (Como podemos ver na figura \ref{fig:centrare_seletor_alterado}).

\begin{figure}
    \centering
    \includegraphics[width=1\textwidth]{imgs/CENTRARE/instanciaNormal.png}
    \caption{Nesta imagem podemos ver o workflow CENTRARE e o seletor de atividade inicial na parte superior da imagem. O seletor contém o escrito "Selecione uma atividade", indicando que ele não foi alterado e a atividade que está sendo mostrada é atividade inicial original deste workflow.}
    \label{fig:centrare_seletor_normal}
\end{figure}

\begin{figure}
    \centering
    \includegraphics[width=1\textwidth]{imgs/CENTRARE/instanciaAlterada.png}
    \caption{Nesta imagem podemos ver o workflow CENTRARE e o seletor de atividade inicial na parte superior da imagem. O seletor contém o escrito "Cirurgia de Nariz", indicando que o usuário selecionou a atividade com este nome para ser focada e utilizada como atividade inicial. Como podemos ver pela tabela, temos todas as atividades "Cirurgia de nariz" representadas como instâncias do workflow.}
    \label{fig:centrare_seletor_alterado}
\end{figure}

A árvore de atividades também teve de ser alterada, com a criação de novos tipos de atividades: Atividades pai, ou atividades anteriores. Neste tipo de atividade (Identificados pela cor azul na figura \ref{fig:centrare_tree_normal_altered}), não é possível executar novas atividades, sendo existentes apenas por motivos de disponibilização de informações pertinentes à execução atual.

Foi necessário a criação deste novo tipo de atividade para disponibilizar informações de atividades anteriores para o usuário, já que as informações podem ser pertinentes para o usuário.

Na mesma figura \ref{fig:centrare_tree_normal_altered} temos as informações do paciente como primeira atividade da visualização original, podendo ser necessária para preenchimento de próximas atividades da atividade selecionada "Cirurgia de Nariz".

\begin{figure}
    \centering
    \includegraphics[height=1\textwidth]{imgs/CENTRARE/arvoreNormalEAlterada.png}
    \caption{Imagem demonstrando a árvore de atividades no software original (Esquerda) e a árvore de atividades alterada (Direita). Como podemos ver, a atividade selecionada na esquerda que está no meio do workflow é a mesma atividade selecionada na direita, que agora virou a atividade focada por seleção do usuário. Atividades pai são demonstradas em azul, disponibilizando todas as informações existentes mesmo com a alteração do foco do workflow.}
    \label{fig:centrare_tree_normal_altered}
\end{figure}

\subsubsection{Como funciona}

Quando o usuário seleciona a atividade desejada para ser a atividade inicial, a árvore de atividades é reajustada para que a atividade selecionada seja o foco principal da instância.

Para que isso ocorra, é necessário que a atividade selecionada se torne a atividade inicial, continuando com suas atividades filhas originais mas ganhando uma nova atividade filha: Sua atividade pai. As atividades que eram pais da atividade selecionada se tornam filhas da mesma para que essas informações estejam disponíveis para acesso e para manter a conformidade com a modelagem do BPM já existente.

Para isso, pode-se dizer que ocorre um "giro" na árvore de atividades para a direita, tendo todas as atividades anteriores como atividades filhas da selecionada, mantendo a sub árvore de atividades originais intacta. Podemos ver esta característica na figure \ref{fig:primeira_implementacao}.

\begin{figure}
    \centering
    \includegraphics[width=1\textwidth]{imgs/Implementacoes/primeiraImplementacao.png}
    \caption{Representação da implementação de workflows dinâmicas. Na imagem de cima, podemos ver a implementação original de um workflow genérico que está prestes a ser reconstruído. A atividade em vermelho será utilizada como atividade focada apelo usuário. Com isso, a seta em vermelho representa o "giro" que o workflow faz para que a atividade selecionada vire a primeira atividade do workflow.}
    \label{fig:primeira_implementacao}
\end{figure}

No Flux, esta funcionalidade faz com que atividades pai tenham uma cor diferente na interface do usuário (em azul na figura \ref{fig:centrare_tree_normal_altered}) para que seja claro que as atividades vistas pelo usuário são atividades pai da atividade selecionada. Também é desabilitado a execução de novas atividades a partir das atividades pai.

\subsection{Alterações feitas para múltiplas atividades iniciais}

BPMs não tem suporte para múltiplas atividades iniciais: Um processo deve iniciar e terminar da mesma forma, passando por todo o fluxo de trabalho da organização. \R
Como um mesmo workflow pode ter troca de informações com alguma parte de outros workflows, foi idealizado uma nova funcionalidade para u workflow: Múltiplas atividades iniciais.

Com múltiplas atividades iniciais, é possível agregar diversos workflows que tratam do mesmo laboratório juntos, e cada usuário escolhe em que parte do workflow ele quer (e pode) iniciar um novo trabalho.

Assim, quando se inicia os trabalhos em uma parte diferente de um workflow, e este workflow se comunica com outro BPM, pode-se fazer uma requisição de informações ou execução de atividade por meio do sistema de permissões.

\subsubsection{Como funciona}

Vamos utilizar o exemplo de um pedido de exame em um workflow médico e o recebimento deste pedido de exame e execução do mesmo.
Um médico cria uma instância de paciente, cadastrando-o e seguindo o fluxo de trabalho comum de atendimento.
O laboratórios do hospital também tem seu próprio workflow, onde são cadastrados equipamentos para análise de amostras.
Antes, os workflows ficariam separados, sem comunicação entre eles.

Com o novo recurso implementado, os workflows podem compartilhar atividades como o pedido de exame: O médico tem a permissão de executar o pedido de exame, mas ele não aprova a atividade, quem irá aprovar a atividade é o técnico de laboratório.

O técnico de laboratório recebe uma notificação que a atividade foi executada e pode aprovar ou reprovar a atividade.
Aprovando a atividade, o técnico continua com seu workflow normalmente até o resultado.
Caso o médico tenha permissão de visualizar as atividades entre o pedido de exame e resultado do exame, o médico poderá ver todo o processo de análise.
Caso contrário, o sistema de permissões controla o que o médico poderá ver, que será o pedido de exame e o resultado do exame.

Assim, a árvore de atividades, caso o médico tenha permissão de visualizar apenas o pedido de exame e a execução, fica da maneira representada na figura \ref{fig:segunda_implementacao}

\begin{figure}
    \centering
    \includegraphics[width=1\textwidth]{imgs/Implementacoes/segundaImplementacao.png}
    \caption{Representação da implementação de múltiplas atividades iniciais. Neste exemplo, temos a instância de pacientes em preto e em vermelho temos o workflow do técnico de laboratório. Podemos ver que as atividades em preto e vermelho são compartilhadas, e apenas o técnico tem permissão de visualização das atividades entre o pedido de exame e o resultado do pedido.}
    \label{fig:segunda_implementacao}
\end{figure}

% Workflow grande feito no Flux