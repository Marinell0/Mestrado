\selectlanguage{english}

\begin{abstract}
    Systems for collecting, storing and analyzing data are increasingly necessary with each technological evolution. The biomedical and laboratory areas are no exception, and there are specialized systems for this type of function: LIMS (Laboratory Information Management System).

    Workflow-based LIMS are systems that use an organization's business process model (BPM), which are visual representations of how business activities are performed to improve efficiency and optimize organizational processes. These systems have the advantage of making this process available so that users can perform their jobs efficiently, orderly and quickly, keeping information in the system for other users to access.

    However, these complex systems store all the information of the organizations that use them and, as there is a lot of data being stored, the system needs to make them available in the most intuitive way possible.
    This work proposes implementations that change the business process model built in the system to centralize the user's focus on an activity, facilitating access to the information required at the time.

    The ability to share information between workflows was also implemented so that teams can communicate entirely through the system, reducing the incidence of errors when communication between teams is necessary.

    These implementations are in being used in several systems with benefits for users. In particular, we mention the workflow CTTX, which was unified with the Good Laboratory Practices (GLP) workflow for sharing information on the equipment used. We also mention the dynamic visualization of the CENTRARE workflow, used in the hospital "Hospital da Baleia", to facilitate the search for information within the workflow.
\end{abstract}

\selectlanguage{Portuguese}