\subsection{Múltiplos pontos de início de um workflow}

% Separação de setores do workflow em categorias de usuários

% O que isso traz para os BPMs?

% Isso já é possibilitado pelos BPMs?

Nos BPMs, não existem múltiplas atividades iniciais, já que um processo consiste de um início, e só a partir dele temos como dar progresso ao modelo de negócios. Com isso, a adição de múltiplas atividades iniciais que podem ser iniciadas por pessoas diferentes adiciona uma poderosa ferramenta a modelagem dos workflows.

% O que isso traz para os usuários

A utilização de múltiplos inícios para BPMs facilita o compartilhamento de partes de processos de negócio que podem se repetir entre diferentes BPMs. As atividades repetidas podem ser reutilizadas em cada BPM e compartilhadas caso as informações sejam pertinentes para os BPMs envolvidos.

Com a utilização do sistema de permissões presente em um LIMS, podemos utilizar o compartilhamento de atividades para bloquear o acesso de um usuário "A" à atividades que serão realizadas pelo outro usuário "B" e fazendo com que a execução desta atividade libere atividades pertinentes ao usuário "B". Quando o usuário "B" terminar de executar todas as atividades pertinentes ao fluxo do seu trabalho, uma nova atividade compartilhada entre os dois usuários pode ser executada pelo usuário "B", que será disponibilizada para o usuário "A", dando informações do trabalho realizado.

Um exemplo disso é em um pedido de exame de sangue de um médico para um laboratório, que irá examinar a amostra e dar um resultado ao médico. Este médico só poderá continuar o processo de atendimento quando o técnico de laboratório tiver terminado de realizar todas as análises da amostra e executando uma atividade de resultado de análise, que o médico tem acesso, tendo assim a troca de informações entre médico e técnico de laboratório e melhorando a integração entre os dois processos.

% O que isso traz para os LIMS

A integração desta nova ferramenta a um LIMS melhora muito a eficiência e integração das atividades de um modelo de negócios com o outro, otimizando a troca de informações entre dois fluxos de trabalho. O médico pode receber uma notificação quando o exame ficar pronto, um técnico recebe notificação quando uma amostra chega e pode até ser feita a otimização deste caminho, com a amostra indo direto para uma máquina analisar.

Isso também facilita a utilização do software pelos usuários, já que fica mais intuitivo onde e quando o usuário deve entrar dentro da interface para acessar os dados que são de seu interesse.

% Porque isso é necessário?

\subsection{Usos}

É possível vincular atividades de diferentes BPMs umas as outras por meio do compartilhamento da atividade entre usuários, utilizando o sistema de permissões do LIMS para disponibilizar informações pertinentes para cada usuário. Separadamente, os workflows não podem comunicar entre si por serem modelados separadamente utilizando o BPM, mas quando esta nova ferramenta, unindo múltiplos workflows em uma mesma modelagem, podemos ter uma visão de todos os workflows vinculados, além de receber notificações de atividades concluídas (Exemplo: Calibração de um equipamento) e fazer fluxos de trabalho movidos a eventos: Um cientista pede a análise de uma mostra a um técnico de laboratório, que por sua vez recebe a notificação que existe uma atividade a ser feita. Quando concluída, o cientista recebe uma notificação e uma nova atividade está disponibilizada para ser feita.