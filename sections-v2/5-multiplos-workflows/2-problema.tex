\subsection{Problema}

É comum que as empresas utilizem múltiplos BPMs para gerenciar processos de diferentes setores e funcionários.
No entanto, a utilização de múltiplos BPMs pode criar silos de informação e dificultar a comunicação entre os diferentes setores da empresa. Isso pode levar a uma falta de visibilidade e compreensão sobre o desempenho geral dos processos de negócio da organização.

Uma forma de melhorar o desempenho da organização é promover a comunicação entre os diferentes BPMs utilizados pela empresa. Isso pode ser alcançado por meio da integração dos sistemas, permitindo que as informações sejam compartilhadas entre eles e proporcionando uma visão mais completa dos processos de negócio da organização.

Além disso, a comunicação entre os BPMs pode ajudar a identificar possíveis conflitos ou ineficiências nos processos, permitindo que a organização tome medidas para otimizá-los. A comunicação também pode ajudar a garantir que os processos de negócio estejam alinhados aos objetivos estratégicos da organização.

Quanto dois BPMs têm passos de processo iguais, isso significa que eles estão lidando com o mesmo tipo de atividade ou tarefa. Ao compartilhar esses passos, a organização pode reduzir o esforço duplicado de modelar e implementar o mesmo processo em sistemas separados.

Atividades compartilhada em um LIMS permite que informações relevantes sobre amostras, resultados de testes e outras informações sejam compartilhadas entre diferentes departamentos ou locais de um laboratório. Isso pode ajudar a melhorar a comunicação e colaboração entre as equipes, permitindo que eles trabalhem de forma mais eficiente e coordenada.

Além disso, o envio de eventos por meio da atividade compartilhada em um LIMS (por meio de notificações) pode permitir que as equipes monitorem e respondam rapidamente a eventos, como pedidos de exame e calibrações de equipamento. Isso pode ajudar a garantir que os processos de laboratório sejam executados de forma eficiente e que os prazos de entrega sejam cumpridos.