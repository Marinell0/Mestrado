\subsection{Troca de informações}

% Workflows com muitas pessoas utilizando

Quando uma organização dispõe de muitos funcionários com funcionalidades diferentes, são modelados BPMs diferentes para cada um e, muitas vezes, é necessário a comunicação entre eles~\cite{Holbein1996AOrganisations}. Com isso, algum tipo de implementação que disponibiliza a troca de informações entre atividades de diferentes BPMs modelados dentro de um LIMS ajuda na cooperação entre usuários de diferentes workflows.

% Troca de informações entre essas pessoas


% O que a troca de informações possibilita

% Movido para a seção que descreve a funcionalidade
% É possível vincular atividades de diferentes BPMs umas as outras por meio do compartilhamento da atividade entre usuários, utilizando o sistema de permissões do LIMS para disponibilizar informações pertinentes para cada usuário. Separadamente, os workflows não podem comunicar entre si por serem modelados separadamente utilizando o BPM, mas quando esta nova ferramenta, unindo múltiplos workflows em uma mesma modelagem, podemos ter uma visão de todos os workflows vinculados, além de receber notificações de atividades concluídas (Exemplo: Calibração de um equipamento) e fazer fluxos de trabalho movidos a eventos: Um cientista pede a análise de uma mostra a um técnico de laboratório, que por sua vez recebe a notificação que existe uma atividade a ser feita. Quando concluída, o cientista recebe uma notificação e uma nova atividade está disponibilizada para ser feita.

% Facilidades para usuários

Uma visão macro do workflow, ou seja, uma visão de todos os workflows em uma mesma estrutura e em uma mesma interface, ajuda por exemplo, um gerente de laboratório a visualizar quais trabalhos estão sendo feitos e obter dados como quais atividades devem ser feitas com mais frequência, onde estão as ineficiências do laboratório e como melhorá-las.

O compartilhamento de informações entre funcionários da organização que implementa o LIMS com compartilhamento de informações entre atividades ganha maior eficiência por ter todos os dados necessários para o seu trabalho dentro da mesma interface já utilizada.

A organização é otimizada com a maior integração do software nos afazeres diários dos funcionários, além de ter uma maior segurança dos dados, já que ficarão armazenados com o mesmo sistema de segurança para todas as funções.

% Faciliades para consumo dos dados por analise

A comunicação entre workflows permite que exista uma auditoria dentro do LIMS para cada troca de informações, facilitando a análise do processo como um todo, que é o objetivo da implementação de um BPM para modelar um processo de negócios. Temos o tempo de resposta entre workflows distintos dentro da empresa, dados sobre a comunicação que poderiam ser perdido, maior automação dos processos como um envio de uma amostra diretamente para o usuário interessado ou até mesmo para a própria máquina que lê os dados e já os processa.

Esta comunicação entre workflows ocorre compartilhando informações de uma mesma atividade entre múltiplas instâncias de BPMs diferentes, unidos em um mesmo workflow com múltiplas atividades inicial. Isto é, uma mesma atividade A executada pode ser acessada por todos os fluxos de trabalho que contém aquela atividade sendo compartilhada entre elas.

Utilizando o sistema de permissões utilizado no LIMS Flux \R e a reestruturação feita no software para permitir que atividades possam conversar entre si, pode ser feita uma implementação de troca de informações e compartilhamento de trabalhos com diferentes setores de uma empresa, tornando uma ferramenta poderosa para que múltiplos workflows de um mesmo laboratório possam se juntar, tendo comunicação entre os usuários e aumentando a integração.