% O que é o Flux

O LIMS Flux é um sistema de gerenciamento de informações de laboratório (LIMS) que permite que laboratórios gerenciem e rastreiem seus processos e dados de amostras. O LIMS Flux é um software baseado em nuvem que pode ser acessado de qualquer lugar, a qualquer momento, através de um navegador da web.

O sistema é altamente configurável e pode ser personalizado para atender às necessidades específicas de cada laboratório. Ele é projetado para ajudar laboratórios a gerenciar fluxos de trabalho, desde o registro de amostras até a geração de relatórios finais.

\subsection{Como funciona}

O Flux utiliza BPMs como base de modelagem para desenvolvimento dos workflows dentro do software. Com o treinamento necessário e o conhecimento técnico de modelagem de BPMs, um administrador laboratorial pode utilizá-lo para criar um workflow seguindo o BPM modelado para a organização, personalizando o fluxo de trabalho de maneira a se integrar com o laboratório.

O Flux utiliza o conceito de BPM na construção de seus workflows, definindo cada parte do fluxo de trabalho como atividades e como atributos as informações dentro de cada atividade. Com isso, pode-se construir qualquer tipo de workflow dentro do Flux, podendo abstrair BPMs para serem implementados.

Ele pode ser considerado um meta LIMS: O software pode ser utilizado para criação de workflows (fluxos de trabalho) com atividades e atributos que seguem os padrões de um laboratório ou telemedicina, conseguindo automatizar a entrada de informações da atividade e integrar com o restante do laboratório independente de outros softwares utilizados.

A integração de outros softwares é uma capacidade importante que permite que diferentes aplicativos e sistemas funcionem juntos, trocando informações de forma automática para melhorar a eficiência do LIMS. No caso do Flux, ele atinge essa funcionalidade com a funcionalidade de executar qualquer tipo de programa em qualquer linguagem por meio de plugins dentro do workflow criado.
Sendo assim, o Flux pode ser integrado com todo o laboratório, não sendo necessário que algum administrador do software implemente alguma iteração e disponibilize para os usuários.

% Porque o Flux foi utilizado

\subsection{Utilização do Flux}

O Flux foi utilizado pela facilidade na modelagem de workflows variados utilizando BPM. Tendo em mente os objetivos deste trabalho, o Flux foi modificado para atender às demandas para resolução dos mesmos, como alteração da interface a depender do usuário que está utilizando do software, apresentando informações mais rapidamente para os usuários, diminuindo o tempo de busca dentro do programa e também a troca de informações entre diferentes BPMs, unificando a modelagem de diferentes tipos de workflows em um grande BPM com diferentes atividades iniciais.

% Como o Flux utiliza Business Process Model

Com a utilização de BPMs de maneira modular feito no Flux, há uma maior flexibilidade na modelagem dos processos dentro do software, diminuindo os custos de desenvolvimento de software já que qualquer pessoa que tem o conhecimento do próprio software e de construção de BPMs consegue desenvolver um fluxo de trabalho de um modelo de negócios.

Isso diminui (mas não elimina) alguns dos problemas que envolvem LIMS que implementam BPMs como o custo elevado pela complexidade de desenvolvimento do programa, pois ainda necessita de pessoas que saibam modelar os processos mas não necessariamente que sejam desenvolvedores de software.

% Como o Flux revela o problema de todos os LIMS com BPM

Com o Flux, construímos BPMs complexos para serem implementados no software, demonstrando a necessidade de melhorias na obtenção de dados para agilizar o trabalho dos usuários. Os BPMs complexos que foram utilizados são utilizados hoje por organizações que utilizam o Flux como ponto central nos trabalhos feitos.

BPMs, sendo apenas uma maneira de descrever fluxos de trabalho e modelagem de processos, não disponibilizam nenhuma forma de filtragem de dados, ficando a dever do LIMS de implementar algum tipo de recurso para disponibilização de informações aos usuários.

Quando o fluxo de trabalho é muito grande, tomamos tempo do usuário, fazendo-o procurar dentro do fluxo de trabalho, dentre inúmeras informações disponíveis, a informação que ele deve encontrar. Isso faz com que o usuário perca tempo em forma de inúmeros cliques dentro de uma interface que muitas vezes não é intuitiva~\cite{OvationIncNeeded:LIMS}.

Com múltiplas pessoas tendo diferentes funções dentro do mesmo workflow, é muito improvável que todas elas estejam fazendo a mesma função e necessitando das mesmas informações. A interface, então, deve ser alterada para que cada usuário tenha uma visão diferente quando está utilizando o LIMS para que, com uma experiência personalizada, haja uma diminuição do tempo de procura das informações buscadas.

Para a telemedicina, isso é de suma importância pois os médicos, que podem se tornar usuários de um LIMS, não gostam de diminuir a eficiência dos seus trabalhos aprendendo uma ferramenta nova e modelos de trabalho novos sendo que o modelo existente para armazenamento de dados já funciona para eles, então o LIMS deve ser intuitivo, personalizável e disponibilizar as informações necessárias com a menor resistência possível para o usuário.