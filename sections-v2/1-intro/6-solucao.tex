\subsection{Solução}

% Qual LIMS foi utilizado

Para que a personalização da interface do usuário seja integrada com BPMs, foi envisionado uma nova maneira de execução de fluxo de trabalho de BPMs dentro de um LIMS. Para isso, foi utilizado o LIMS Flux (mais explicado na seção \ref{sec:flux}) para implementação de workflows dinâmicos.

Workflows dinâmicos permitem que o usuário centralize sua visão em uma atividade central dentro desejada, sendo particularmente útil em organizações que possuem workflows onde as atividades podem ser complexas e envolver várias etapas.

A funcionalidade de centralização de atividade dentro de um workflow pode ser utilizada pelos usuários para selecionar a atividade desejada para obter dados requeridos de maneira mais rápido, atendendo às necessidades específicas daquele usuário de maneira pontual. Isso permite que cada pessoa se concentre nas atividades que são mais relevantes para suas tarefas no momento da utilização do software, personalizando seu uso.

% Solução quanto ao acesso de informações resolvida com o trabalho

% Quem isso afeta e porque é importante?

% Como foi feito

% Relação com BPM

% Utilização no Flux