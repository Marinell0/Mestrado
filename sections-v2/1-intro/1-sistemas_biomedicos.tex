\subsection{Sistemas de informações modernos}

% Big data 
% Grande número de informações

Nos últimos anos, temos observado um aumento exponencial na quantidade de dados gerados em todo o mundo, provenientes de diversas fontes como redes sociais, dispositivos móveis, sensores IoT (Internet of Things) e transações financeiras.
Esses dados são muito importantes para empresas e organizações de todos os setores, pois fornecem informações valiosas sobre o comportamento do consumidor, as tendências do mercado, desempenho dos negócios e até mesmo dados internos como utilização de recursos dentro da empresa.

No entanto, a gestão desses dados pode ser um desafio, pois eles estão dispersos em diversos locais, podendo ter sido salvos em diferentes formatos. Além disso, é essencial garantir a segurança e a privacidade dessas informações, especialmente quando se trata de dados confidenciais de clientes, pois podem ser dados sensíveis que não podem ser disponibilizados para o público.

A análise de dados também é uma parte crítica desse processo, pois permite que as empresas compreendam e usem seus dados para tomar decisões de negócios mais informadas. Com a análise de dados, as empresas podem identificar tendências e padrões em seus dados, permitindo que elas otimizem seus processos e melhorem a experiência do cliente.

Para lidar com esses desafios, os sistemas de controle de acesso, coleta e análise de dados tornaram-se cada vez mais importantes. Esses sistemas permitem que as empresas gerenciem seus dados com segurança e eficiência, garantindo que apenas pessoas autorizadas possam acessá-los e que as informações sejam coletadas e armazenadas de maneira adequada.

Para áreas de telemedicina, laboratorial e biomédica, este tipo de sistema é essencial para o armazenamento, processamento e segurança das informações coletadas, integrando-se ao ambiente onde foi implementado para aumentar a eficiência dos trabalhos e tornar possível a análise de todos os dados coletados nos trabalhos feitos.

Os chamados sistemas biomédicos permitem que pesquisadores, médicos e profissionais da saúde gerenciem e analisem grandes quantidades de dados de pacientes e estudos clínicos com mais eficiência e segurança, diminuindo a incidência de erros que podem ocorrer~\cite{Sun2021LaboratoryEfficiency}.

\subsection{Sistemas biomédicos}

Sistemas biomédicos trazem muitos benefícios à área biomédica, como a melhoria de segurança, garantindo acesso aos dados apenas para pessoas autorizadas, aumento de eficiência por disponibilizar a coleta de dados automatizada quando o sistema está integrados com equipamentos, reduz erros de entrada de usuários por garantir que os dados estejam corretamente formatados e também melhoram a gestão de recursos devido a análise de dados disponibilizada. Eles podem ajudar, por exemplo, instituições médicas, que utilizam esta tecnologia para garantir que os recursos estejam sendo usados de maneira eficiente e econômica.

Para isso, é fundamental que se tenha uma maneira eficiente de gerenciar e armazenar essas informações, disponibilizando a integração entre variados tipos de equipamento como sensores de temperatura, monitores cardíacos e medidores de pressão arterial. Por este tipo de funcionalidade ser altamente procurada por laboratórios, surgiram sistemas especializados para que a integração laboratorial seja feita de maneira rápida e simples, os sistemas de gerenciamento de informações de laboratório, ou LIMS \cite{Skobelev2011LaboratoryLaboratory}.

% Coleta de informações

% Tempo gasto ao salvar dados~\cite{Sinsky2016}

% Armazenamento e organização

% Segurança

% Gestão de laboratórios é extremamente necessária no mundo tecnológico em que vivemos hoje. Aumento de eficiência, automação de processos, gestão de tempo e funcionários são algumas das funcionalidades que são procuradas atualmente em diversos laboratórios~\cite{sun2021laboratory}. Para esse tipo de gestão, é necessário algum meio de formalizar fluxos de trabalhos para que sejam de fácil entendimento e reproduzíveis. 

% Para isso, muitas soluções foram pensadas para atacar este problema, tanto na parte de armazenamento de dados quanto em segurança de dados \R. Como exemplos de soluções feitas para armazenamento de dados laboratoriais, temos \R, \R e \R.