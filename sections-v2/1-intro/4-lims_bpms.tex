\subsection{LIMS baseados em BPM}

% Utilização de LIMS junto a BPMS

LIMS baseados em BPMs juntam a modelagem de fluxos de trabalho com a utilização do BPM e a gestão de dados em um LIMS, ajudando a melhorar a eficiência e qualidade dos processos de laboratório e permitindo que os usuários configurem, gerenciem e monitorem os processos de forma mais eficiente. Isso é feito através da automação de fluxos de trabalho, permitindo que as tarefas sejam atribuídas aos usuários corretos, no momento certo e com a quantidade certa de informações. \R

Além disso, o uso de BPM em um LIMS permite a integração de diferentes sistemas e aplicativos, facilitando a comunicação entre as equipes e melhorando a colaboração, já que todo o processo de negócio das equipes estará modelada dentro do LIMS. A padronização dos processos também ajuda a garantir a qualidade e precisão dos dados gerados pelo laboratório.

% Exemplos de uso ultimamente

\subsubsection{Vantagens}

Os BPMs trazem melhorias significativas com sua utilização em um LIMS, como a padronização de processos que serão executados, reduzindo a variabilidades nos resultados e melhorando a confiabilidade das análises, além de facilitar a automatização de tarefas repetitivas e padronizadas, o que diminui ainda mais os erros e aumentam a produtividade.

Com a definição do fluxo de negócios implementada em um LIMS, é possível também fazer a otimização deste processo, já que o BPM permite a identificação de gargalos e pontos de melhoria nos processos de laboratório, permitindo uma melhoria contínua.

Um LIMS implementado em laboratórios de saúde pode ser utilizado para gerenciamento de coleta, processamento e análise de amostras de pacientes. Quando um BPM está implementado neste mesmo LIMS, há a definição dos processos envolvidos na realização dos mesmos por meio de diagrama de fluxos, facilitando sua otimização e colaboração entre as equipes de laboratório e outros departamentos.

Com a definição do processo e gerenciamento dos dados pelo LIMS, diminui-se a quantidade de erros que podem acontecer na entrada de dados pelos usuários do software, já que se pode ter uma integração grande com o software de armazenamento de dados diretamente com uma máquina que processa amostras médicas, por exemplo.

Também há a maior facilidade de seguir regulamentações. Os laboratórios são regulamentados por várias agências reguladoras, como a FDA (Food and Drug Administration), ANVISA (Agência Nacional de Vigilância Sanitária) e ISO (International Organization for Standardization). Essas agências estabelecem requisitos rigorosos para garantir a qualidade e segurança dos produtos e serviços fornecidos pelos laboratórios.

A utilização de um LIMS que implementa BPM aumenta a rastreabilidade de amostras desde o momento que são recebidas até o momento em que são descartadas, melhora o gerenciamento de documentos como procedimentos operacionais padrão (SOPs) e também servem para auditoria dos processos e validação dos resultados por salvarem todas as informações dentro dos seus bancos de dados, garantindo que eles sejam precisos e confiáveis e atendam aos requisitos regulatórios.

% Qual a vantagem e desvantagem com LIMS que não utilizam BPMs

\subsubsection{Desvantagens}

Algumas das desvantagens de uma implementação de LIMS com BPM juntos é o preço de implementação, complexidade para o usuário, personalização do BPM para o LIMS e falta de flexibilidade a depender da implementação do software.

O preço vem da complexidade de implementação das duas tarefas: tanto do LIMS quanto do BPM \R. Com a complexidade de implementação, vem também a complexidade para os usuários entenderem sua utilização em uma interface intuitiva \R.

A integração entre BPM e LIMS também pode gerar muitas dificuldades no quesito de personalização de interface e implementação de um BPM que seja muito complexo para o LIMS que esteja tentando implementá-lo. Caso este seja o caso, o LIMS deve ser alterado para que este processo possa ser implementado corretamente. Com isso, há uma maior dificuldade de flexibilidade do software, já que alguns softwares não suportam que sejam feitas muitas alterações para que um BPM complexo possa ser corretamente modelado dentro do sistema.