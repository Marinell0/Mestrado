\subsection{LIMS}

% O que são LIMS

LIMS (Laboratory Information Management System) é um tipo de sistema de gerenciamento de informações de laboratório que permite o gerenciamento e controle de todas as informações e processos de laboratório em um único sistema integrado. Ele ajuda os laboratórios a automatizar e gerenciar tarefas complexas como coleta, armazenamento e análise de dados, gerenciamento de amostras e rastreabilidade, gerenciamento de estoque e inventário, além de garantir a conformidade regulatória.

% Porque são relevantes de acordo com os pontos anteriores

A utilização do LIMS oferece diversos benefícios, como maior eficiência operacional, redução de erros manuais, melhoria da qualidade dos dados, automação de fluxos de trabalho e melhoria da colaboração e compartilhamento de informações~\cite{Key2011LIMS:Systems}. O LIMS também ajuda na rastreabilidade de amostras e resultados, permitindo que os usuários identifiquem facilmente a origem dos dados e possam rastrear as informações em caso de necessidade \R.

% Como são utilizados no setores: Médicos, Laboratoriais

Nos setores médicos, esses sistemas são usados para gerenciar registros eletrônicos de pacientes, permitindo que os médicos e profissionais de saúde acessem e atualizem informações em tempo real, o que ajuda na tomada de decisões clínicas mais precisas e rápidas. Esses sistemas também são usados para gerenciar e rastrear amostras de pacientes, bem como para gerenciar o estoque de medicamentos e suprimentos médicos.

Nos setores laboratoriais, os softwares LIMS são usados para gerenciar e rastrear amostras de pacientes, bem como para gerenciar o fluxo de trabalho e o inventário de reagentes e equipamentos. Esses sistemas ajudam a garantir a rastreabilidade e a integridade das amostras, a melhoria da qualidade dos dados e a conformidade com as regulamentações.

Além disso, os LIMS são utilizados em laboratórios de pesquisa para gerenciar grandes volumes de dados e informações de experimentos, garantindo a precisão, segurança e a integridade dos dados. Esses sistemas também ajudam na colaboração entre os membros da equipe de pesquisa, facilitando o compartilhamento de informações e resultados.

LIMS podem ser feito de diversas maneiras para integrar as atividades de um laboratório ao software, e uma dessas maneiras pode ser utilizando BPMs.

% Como podem ser utilizados em tudo

% A utilização de um LIMS traz grandes vantagens para laboratórios que o utilizam, como a automação de processos e coleta e armazenamento de dados~\cite{Key2011}. Os LIMS baseados em BPMs usam a modelagem em BPM para abstrair atividades (uma ação dentro do BPM) que podem ser repetidas ou não, como em experimentos que devem ser refeitos ou atendimento a pacientes. 