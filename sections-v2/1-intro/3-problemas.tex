\subsection{Problemas com LIMS hoje}

% Introdução ao problema com LIMS hoje

Os LIMS hoje, mesmo com todos os benefícios citados acima, ainda sofrem de problemas que limitam a sua implementação dentro de ambientes médicos e laboratoriais, como a personalização da interface, custo de implementação, manutenção e suporte dentro da organização~\cite{Avery2000ProductGuide., 2018CommonAstrix}.

O software pode não ser tão personalizável porque, em grande parte dos casos, eles são projetados para atender a um conjunto específico de requisitos regulatórios e padrões de indústria, além de envolverem a integração de muitos componentes diferentes como banco de dados, interface do usuário e sistemas de instrumentação que podem dificultar a personalização do mesmo~\cite{Tomlinson2022AOperations}.

Outro problema que dificulta a personalização de um LIMS é a necessidade de manter a validação e a conformidade regulatória, já que cada personalização do sistema deve passar por uma validação vigorosa para garantir que qualidade e integridade da entrada de dados não foram comprometidas.
Com isso, há um aumento da complexidade de implementação e um aumento do custo, já que o software se torna complexo, necessitando de especialistas para a construção do mesmo.

% Problema na visualização de atividades

% Problema na aceleração de acesso à informação

A falta de personalização de um LIMS pode levar a problemas adicionais relacionados à visualização e acesso à informação. Quando um sistema LIMS não é personalizado para atender as necessidades específicas do laboratório, as informações podem ser apresentadas de maneira desorganizada, o que dificulta o acesso e a visualização dos dados relevantes e diminui a aceitação dos usuários para utilizar o software. Por exemplo, um laboratório que realiza vários tipos de testes pode ter dificuldades para visualizar informações específicas de um teste em particular, se o sistema LIMS não estiver configurado adequadamente para exibir essas informações de maneira clara e organizada.

% Porque isso deve ser melhorado?

Além disso, a falta de personalização pode tornar a navegação pelo sistema LIMS mais difícil, o que pode levar a erros ou omissões no registro de informações e dificultar a interpretação dos dados. Se os usuários não conseguirem acessar rapidamente as informações que procuram, eles podem inadvertidamente inserir dados incorretos ou perder informações críticas. Isso pode levar a erros na análise dos resultados dos testes e na tomada de decisões clínicas.

Portanto, a personalização adequada de um sistema LIMS é essencial para garantir que as informações sejam apresentadas de maneira clara e organizada e que os usuários possam acessá-las facilmente~\cite{Tomlinson2022AOperations}. A interface do usuário deve ser projetada para ser intuitiva e fácil de usar, permitindo que os usuários naveguem facilmente pelas informações relevantes. Além disso, o sistema deve ser capaz de exibir informações personalizadas para diferentes usuários ou grupos de usuários, de modo que cada pessoa possa acessar as informações relevantes para suas tarefas específicas.

% Quem liga pra esse problema?

Como exemplo, um LIMS com interface dinâmica implementado na área hospitalar para assistir na execução de atividades entre a equipe médica e a equipe laboratorial pode beneficiar tanto médicos quanto outros profissionais de saúde que trabalham com laboratórios pois terão acesso a informações relevantes - como pedidos de exame, calibração de equipamento - que ajudarão a orientar o tratamento de seus pacientes de maneira mais rápida e eficiente.

Para que isso seja possível, é necessário que ocorra uma modelagem dos passos a serem executados pelos usuários durante a utilização do LIMS, melhorando a compreensão de onde e como aperfeiçoar a visualização de informações e deixar mais intuitiva a interface. Essa modelagem pode ser feita de diversas maneiras, e uma dessas maneiras é utilizando BPMs.