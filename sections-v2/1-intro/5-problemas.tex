\subsection{Problemas com LIMS hoje}

% Introdução ao problema com LIMS hoje

Os LIMS hoje, mesmo com todos os benefícios como automatização de tarefas, redução de erros manuais, aumento da produtividade e rastreamento de dados, ainda sofre de grandes problemas que, muitas vezes, limitam a sua implementação dentro de ambientes médicos e laboratoriais.

Os problemas vão desde a personalização de interface até o custo de implementação, manutenção e suporte dentro da organização~\cite{Avery2000ProductGuide., CommmonAstrix}.
Os LIMS não são tão personalizáveis porque, em grande parte dos casos, eles são projetados para atender a um conjunto específico de requisitos regulatórios e padrões de indústria, além de envolverem a integração de muitos componentes diferentes como banco de dados, interface do usuário e sistemas de instrumentação que podem dificultar a personalização do mesmo.

Outro problema que dificulta a personalização de um LIMS é a necessidade de manter a validação e a conformidade regulatória, já que cada personalização do sistema deve passar por uma validação vigorosa para garantir que qualidade e integridade da entrada de dados não foram comprometidas.
Com isso, há um aumento da complexidade de implementação e um aumento do custo, já que um programa desses passa a ser parte do dia a dia do trabalho na empresa e pode sofrer alterações para manter o fluxo de trabalhos atualizado.

% Problema na visualização de atividades

% Problema na aceleração de acesso à informação

A falta de personalização de um LIMS pode levar a problemas adicionais relacionados à visualização e acesso à informação. Quando um sistema LIMS não é personalizado para atender às necessidades específicas do laboratório, as informações podem ser apresentadas de maneira desorganizada, o que dificulta o acesso e a visualização dos dados relevantes. Por exemplo, um laboratório que realiza vários tipos de testes pode ter dificuldades para visualizar informações específicas de um teste em particular, se o sistema LIMS não estiver configurado adequadamente para exibir essas informações de maneira clara e organizada.

% Porque isso deve ser melhorado?

Além disso, a falta de personalização pode tornar a navegação pelo sistema LIMS mais difícil, o que pode levar a erros ou omissões no registro e interpretação dos dados. Se os usuários não conseguirem acessar rapidamente as informações de que precisam, eles podem inadvertidamente inserir dados incorretos ou perder informações críticas. Isso pode levar a erros na análise dos resultados dos testes e na tomada de decisões clínicas.

Portanto, a personalização adequada de um sistema LIMS é essencial para garantir que as informações sejam apresentadas de maneira clara e organizada e que os usuários possam acessá-las facilmente. A interface do usuário deve ser projetada para ser intuitiva e fácil de usar, permitindo que os usuários naveguem facilmente pelas informações relevantes. Além disso, o sistema deve ser capaz de exibir informações personalizadas para diferentes usuários ou grupos de usuários, de modo que cada pessoa possa acessar as informações relevantes para suas tarefas específicas.

% Quem liga pra esse problema?

Como exemplo, um LIMS com interface dinâmica implementado na área hospitalar para assistir na execução de atividades entre a equipe médica e a equipe laboratorial pode beneficiar tanto médicos quanto outros profissionais de saúde que trabalham com laboratórios pois terão acesso a informações revelantes que ajudarão a orientar o tratamento de seus pacientes de maneira mais rápida e eficiente.